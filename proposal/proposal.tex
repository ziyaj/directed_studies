\documentclass[11pt,letterpaper]{article}

%%%%%%%%%%%%%%%%%%%%%%%%%%%%%%%%%%%
\pagestyle{plain}
%%%%%%%%%% EXACT 1in MARGINS %%%%%%%%%%%%%
\setlength{\textwidth}{6.5in}
\setlength{\oddsidemargin}{0in}
\setlength{\evensidemargin}{0in}
\setlength{\textheight}{8.5in} 
\setlength{\topmargin}{0in}
\setlength{\headheight}{0in}
\setlength{\headsep}{0in}
\setlength{\footskip}{.5in}
%%%%%%%%%%%%%%%%%%%%%%%%%%%%%%%%%%%%

\title{CPSC 448 - Directed Studies Proposal}
\author{Ziyang Jin \thanks{Ziyang Jin, 4th year in B.Sc., major in Computer Science.} \\ \texttt{f4a0b@ugrad.cs.ubc.ca}}
\date{}

\begin{document}

\maketitle

I plan to take directed studies in the area of \textit{Graph Drawing} in 2018/19 Winter Term 1, supervised by Prof. William S. Evans.\\

Based on Computational Geometry and Graph Theory, graph drawing deals with data representation problems and has wide applications. Examples include circuit schematics, algorithm animation, and software engineering diagrams\footnote{Giuseppe Di Battista, Peter Eades, Roberto Tamassia, Ioannis G Tollis, Algorithms for drawing graphs: an annotated bibliography, Computational Geometry, Volume 4, Issue 5, 1994, Pages 235-282, ISSN 0925-7721, https://doi.org/10.1016/0925-7721(94)00014-X.}. Graph drawing looks at different representations of graphs and how the drawing of a graph can enhance people's understanding of data. For example, if we look at graphs of human relationships, it can be very complicated, and one way is to group them into clusters, like families, friends, classmates, colleagues, etc. --- this gives us a bigger picture of our relationships that is easier to understand. We can also explore different models of drawing graphs and their limitations. For example, if we put rectangles on a plane, with the rectangle being vertices, and visibility between rectangle determining the edges, what kind of graph can we draw within this constraint? Can we draw any graph? I am interested in understanding the theoretical basis of graph drawing problems and how to find algorithms to produce aesthetically pleasing drawings of graphs.\\

Aligned with the school calendar, the course will officially start on September 4, 2018 and end on December 19, 2018. The course will be taught in forms of lectures, supervised reading, and paper writing. In general, the course will be  divided into 3 stages. In the early stage, I will get familiar with basics of Computational Geometry and Graph Theory. In the middle stage, I will be introduced to some classic algorithms and theories in graph drawing. I will learn techniques of drawing trees, general graphs, planar graphs, directed graphs, and theories behind them. In the final stage, Prof. Evans and I will find an open problem in graph drawing that we are both interested in, and we will conduct some research on it. Upon completion, I will present a paper on new discoveries of this problem. The evaluation of course grade will be based on my progress in the studies and the quality of my final paper.\\

In terms of course material, Prof. Evans will select readings from textbooks, papers, and some of the course notes from researchers in this field, e.g., Prof. Daniel A. Spielman's notes on Spectral Graph Theory\footnote{Here is the course website: http://www.cs.yale.edu/homes/spielman/561/}.

\end{document}
